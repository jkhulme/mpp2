\chapter{Goals}
\label{chap:goals}

To help guide development during the project a number of goals were identified.  Over the course of the project the goals have been refined, new goals have been added and some goals have been dropped.

\section{Previous Goals}
Through research into the problem a number of goals were identified.  After performing a review of the existing software, the goals were refined.  The initial user evaluations also fed into the refinement of the goals of the project.  Some of these goals were completed in the first stage of development, the other goals were carried forward into the second phase of development.

The goals that were completed in the first stage of development are:

\begin{itemize}
\item Improve existing capabilities -- The purpose of this goal was to add to Bio-PEPA's visualisation capabilities and bring these more into line with the other modelling software.  This goal was completed in the first phase of development. The user was given extra control over the appearance of their graphs.  Intensity plotting was also implemented.
\item Visualise the model more intuitively -- The purpose of this goal was to help the biologists to increase their understanding of what the model and the graph is showing them.  This goal was completed in the first phase of development with the implementation of the hierarchical drawing of the model components.
\end{itemize}

The following goals were to be completed during the second phase of development:

\begin{itemize}
\item Provide visualisations that take into account the domain -- The purpose of this goal was also to help the biologists further understand what the model and the graph are showing them. This has been implemented in the second phase of the project.  This goal was merged with animation of the data.
\item Animation of the data -- The purpose of this goal was to make it easier for the biologists to interpret spatial data by showing the species in their results moving through the cell.  This has been implemented: the user can now visualise species moving through a cell.
\item Investigating which visualisation combinations work best -- The purpose of this goal was to try and work out whether there were any combinations of visualisations that the program could offer that the biologists found particularly useful.  This was originally planned to be performed in the second stage of the project, but due to the project focusing on non visualisation features the goal has been removed.
\item Add the ability to annotate the visualisations -- The purpose of this goal is to allow the biologists to add extra supporting information directly onto the visualisation. This goal has been implemented: the user can add annotations to the visualisations.
\item Allow the ability to save and load the program state -- The purpose of this goal was to allow the biologists to complete their work in multiple sessions. This goal has been implemented: users can save and load the program state into files, these can then be emailed to other users who can modify them.
\item Provide a full session history to the user -- The purpose of this goal is to allow the biologists to recover from any mistakes they might make. This has been implemented: the user has a full undo and redo history within the session allowing them to easily correct mistakes.
\item Data Mining -- The purpose of this goal was to have the program provide some level of automatic annotation and setting optimisation to reduce the workload for the biologists.  Due to other features being added that reduce the need for this goal, the goal has now been dropped.  The research for this goal was not lost however as a similar goal was added.  Data mining to identify features is not a goal that should be abandoned either, instead it would be implemented if the development continued.
\item Making meta data accessible -- The purpose of this goal was to display the experiment settings in the \ac{UI}.  This goal has not been completed in this phase of development.  It was pushed back to allow development on real time collaboration and plots as queries features.  It has not been abandoned, instead it would be completed if development was to continue.
\end{itemize}

\section{New Goals}
During the first half of the second phase of development new goals were identified.

\subsection{Data Manipulation and Export}

This goal was added after a meeting with the maintainer of BioDARE~\cite{biodare}.  This web based tool is aimed at results from laboratory experiments, specifically experiments relating to circadian rhythm.

BioDARE is an online biological data repository.  It allows its users to upload their experiment data.  This is then available to all other users.  BioDARE will generate static graphs of the uploaded data.  These plots have little ability to be customised.  Instead users can download the data and customise it in a separate application.

The maintainer explained that he had found that researchers often visualise their data after it has been normalised.  This removes any issues where different species have different scales and makes it easier for the users to interpret the data.

BioDARE can also export the raw data allowing the researchers to use it in other applications if they desire.

This seemed like a very useful feature for users and it was added as a project goal.

\subsection{Time Series Data as a Query}

There has been a lot of research in recent years~\cite{goldin, chakrabarti, popivanov, faloutsos2, bollobas} into using time series data as a query against a database of other time series plots through some similarity measure.

This would be a very useful feature to add to the tool.  By allowing researchers to use a plot as a query they would be able to discover other plots exhibiting similar behaviour.  This could be the same species reacting in the same manner to different species, or a species that is exhibiting similar behaviour.  This could help the biologists discover alternative candidate species for research.

This feature would be most effective if there was an online repository of plots for the biologist to search.  This would also allow them to discover potential collaborators.  Online repositories of data do exist: BioDARE is one of them.  These repositories are unlikely, at the moment, to store the data in a suitable way to allow for efficient search.

Given the potential usefulness of this feature, and the fact that none of the other modelling software had this capability it has been added as a goal.

\subsection{Real Time Collaboration}

An important area where all the reviewed software were lacking is collaboration, real time or not.  The current work flow using the existing software is: perform your analysis, save it, email it to a colleague with additional files and notes and have them open it. They then attempt to work out what the visualisation is showing.  It is a disjointed conversation.

It is not just the modelling software that did not offer collaboration. None of the visualisation software reviewed offered real time collaboration either.

Incorporating real time collaboration was added as a goal to offer researchers a better work flow.
