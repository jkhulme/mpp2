\section{Goals}

\subsection{Previous Goals}
Whilst researching the problem and the existing software, and during the first stage of development a number of goals were identified.

\begin{itemize}
\item Improve existing capabilities
\item Visualise the model more intuitively
\item Visualise closer to the cell level
\item Animation of the data
\item Investigating which combinations work best
\item Add the ability to annotate the visualisations
\item Allow the ability to save and load the program state
\item Provide a full session history to the user
\item Data Mining
\item Making meta data accessible
\end{itemize}

\subsection{New Goals}
During the first half of the second phase of development new goals were identified.

\begin{itemize}
\item Google docs style collaboration.
\item Searchable database of time series plots.
\end{itemize}

\subsubsection{Data Manipulation and Export}

This goal was added after a meeting with the maintainer of BioDARE.  A web based tool that is aimed at results from laboratory experiments, specifically experiments relating to circadian rhythm.

The maintainer explained that he had found that researchers often visualise their data after it has been normalised.  This removes any issues of vastly different scales making it easier for them to interpret. PUT A PICTURE HERE SHOWING THE DIFFERENCE.

BioDARE can also export the raw data allowing the researchers to use it in other applications if they desire.

This seems like a very useful feature for users, and so will be included in the new tool.

\subsubsection{Time Series Data as a Query}

There has been a lot of research in recent years about using time series data as a query against a database of other time series plots.
