\section{Goals}

\subsection{Previous Goals}
Whilst researching the problem, the existing software and during the first stage of development a number of goals were identified.  Some of these goals were completed in the first phase of development:

\begin{itemize}
\item Improve existing capabilities -- This goal was completed in the first phase of development, with the implementation of the intensity plotting and the extra customisation available to the user.
\item Visualise the model more intuitively -- This goal was completed in the first phase of development with the implementation of the hierarchical drawing of the model components.
\end{itemize}

Others were to be completed during the second phase of development:

\begin{itemize}
\item Visualise closer to the cell level - This has been implemented in the second phase of the project.  This goal was merged with animation of the data.
\item Animation of the data - This has been implemented, the user can now visualise species moving through a cell.
\item Investigating which combinations work best - This was planned to be performed in this stage of the project, but due to the new features not all being different visualisations the goal is less relevant now, and has been removed.
\item Add the ability to annotate the visualisations - This goal has been implemented, the user can add annotations to the visualisations.
\item Allow the ability to save and load the program state - This goal has been implemented, users can save and load the program state into files, these can then be emailed to other users who can modify them.
\item Provide a full session history to the user - This has been implemented, the user has a full undo and redo history within the session allowing them to easily correct mistakes.
\item Data Mining - The plan for this has not changed since the first phase.
\item Making meta data accessible - The plan for this has not changed since the first phase.
\end{itemize}

\subsection{New Goals}
During the first half of the second phase of development new goals were identified.

\subsubsection{Data Manipulation and Export}

This goal was added after a meeting with the maintainer of BioDARE.  A web based tool that is aimed at results from laboratory experiments, specifically experiments relating to circadian rhythm.

The maintainer explained that he had found that researchers often visualise their data after it has been normalised.  This removes any issues where different species have different scales and makes it easier for the users to interpret.

BioDARE can also export the raw data allowing the researchers to use it in other applications if they desire.

This seems like a very useful feature for users and will be included in the new tool.

\subsubsection{Time Series Data as a Query}

There has been a lot of research in recent years about using time series data as a query against a database of other time series plots through some similarity measure.

This would be a very useful feature.  It would allow researchers to search through their past work for similar time series data that also shared some species, they then know that different species behave similarly when exposed to another species.

Eventually this database of plots could be a central repository online available to all researchers, allowing users to see if their research overlaps with any other research -- identifying potential collaborators.

\subsubsection{Real Time Collaboration}

An important area that all the reviewed software were lacking in is collaboration, real time or not.  The current work flow using the existing software is do your analysis, save it, email it to a colleague with additional files and notes, have them open it and try and work out what is happening.  It is a disjointed conversation.  None of the tools surveyed offered a more efficient approach.  A new goal for this project is to offer real time collaboration to improve the existing work flow.
