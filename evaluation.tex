\section{Evaluation}

\subsection{First Evaluation}

The first evaluation in the second phase of the project occurred just before the halfway mark.  The group was made up of two people.  One who had taken part in the first evaluation meeting and one person who had no knowledge of the project.

\subsubsection{User Group}

As I did not have a domain expert available I was not able to do insight based evaluation.  I took a more traditional approach.  Before the evaluation I prepared a typical scenario that a user might encounter.  The task was to open a file, annotate it and play the animation, and attach some supporting documentation.  The task was prepared at two levels of breakdown.  One was a paragraph of text at quite a high level.  The second level of breakdown was a step by step instruction of each action to perform.  I then observed them as they attempted the task and offered assistance when required.  Afterwards I gave them a questionnaire to fill in about their experience, afterwords we went through and discussed their answers and any further thoughts that they had.

The task was prepared at two levels to try and gauge how easy the program is too use.  The users were presented the textual description and if they had struggled they would have been given the step by step instructions instead.  The users were able to complete the task from the textual description alone.  This is a good sign that the new tool is usable.

Some issues were encountered:

\begin{itemize}
\item Being unfamiliar with MacOS -- Both users were unable to locate the menu bar as it is not attached to the program as in Windows.  Future evaluations will use Windows.
\item When annotating they were unclear as to what was going to happen when annotating.  For example when annotating the graph with an arrow the user was just left to click twice, with no indication of what would happen.  This has now been fixed, different cursors are used to give feedback to the user that they should click, and rather than just relying on two clicks with no information as to where the arrow is going to point, after the first click (which places the tail of the arrow) an arrow will be drawn that follows the cursor until the second click placing the annotation.
\item Lack of ability to edit, move, or delete annotations -- Once an annotation was placed it was there for good.  The ability to edit annotations was always planned, it just had not been implemented in time.  But the amount of frustration it gave the users was very high.  It was a principle in all three of the design lists that a user should be able to fix mistakes.  Since the evaluation editing and deleting of annotations have been implemented.  This means any mistakes can be corrected.
\end{itemize}


\subsubsection{Personal}

\subsection{Evaluation 2 - Start of Second Semester}
\subsubsection{User Group}
\subsubsection{Personal}

\subsection{Evaluation 3 - End of Second Semester}
\subsubsection{User Group}
\subsubsection{Expert User}
\subsubsection{Personal}
