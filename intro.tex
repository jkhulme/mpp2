\section{Introduction}

Biologists often use computer models to help guide their research as modelling is much cheaper than experimentation.  There are a number of tools available for biological modelling.  These tools typically require a certain level of numerical confidence to create and interpret.  Not all biologists have this numerical confidence.  Some researchers find writing and interpreting models a challenge, this can make them less effective in their research.  It is therefore necessary for the tools they use to help them relate the data to their field by incorporating domain knowledge.

One such tool that can be used for modelling is Bio-PEPA, an extension of the PEPA process algebra.  Bio-PEPA is currently implemented as a plugin for the Eclipse IDE.  Bio-PEPA visualises the model results as time-series graphs.  There is one team, of Src researchers, in particular who use Bio-PEPA and do not have the numerical confidence, as described above, to be comfortable using Bio-PEPA.  This team is the focus for this project.  The purpose of this project was to extend Bio-PEPA's visualisation capabilities to allow the previously mentioned team, and other similar users of Bio-PEPA to more effectively analyse their results.

A significant problem with Bio-PEPA's visualisation capability is that it is difficult to represent spatial change on a time-series graph (as can be seen in Figure~\ref{fig:asrc_intro}).  In Bio-PEPA you can have a species at different locations in the cell, for example, next to the nucleus, next to the cell membrane and throughout the cytoplasm.  The movement of this species through the cell can by modelled by seeing the population of it in each location over time.  This is difficult to visualise on a time series plot as three lines is too abstract.  It requires the use of biological metaphor to be easily interpreted.  In this case using visualisation based on a cell can more intuitively show how the species moves through the cell.  It is this sort of inference that the Src researches find challenging to do with Bio-PEPA currently.

\begin{figure}[h!]
    \centering
    \begin{subfigure}[b]{0.4\textwidth}
        \centering
        \includegraphics[width=0.8\textwidth]{images/asrc_graph_intro.png}
        \caption{Time Series Representation}
        \label{fig:asrc_graph_intro}
    \end{subfigure}
    \begin{subfigure}[b]{0.4\textwidth}
        \centering
        \includegraphics[width=0.5\textwidth]{images/asrc_cell_intro.png}
        \caption{Spatial Representation}
        \label{fig:asrc_cell_intro}
    \end{subfigure}
    \caption{One species at three locations in the cell represented traditionally on a time series graph and also spatially in a cell}
    \label{fig:asrc_intro}
\end{figure}


Over the course of the project the scope has been expanded.  The original objective was to assess which forms of visual representation are most helpful and informative to laboratory science.  At the end of the first project phase the object changed (to reflect that the project was about delivering a finished program the results of many experiments) to be to develop a tool to visualise the results of dynamic, time-series models of intra-cellular behaviour based on biochemical reactions.  This objective was focused on visualisation to aid those researchers who are not numerically confident.  The second phase of the project has added to this objective to also aid interpretation and collaboration.  This change in objective is to make the tool better for all users.

\subsection{Where Does This Tool Fit In?}

In the first stage of the project a review was performed of the features of a number of modelling and visualisation tools.  This review included specialised software aimed at biologists and general software for anybody doing data visualisation.

The software that was reviewed at the start of the project were: Bio-PEPA, Uppaal, V-Cell, Cell-O, Copasi, Cell Designer and WEAVE.  Bio-PEPA, V-Cell, Cell-O, Copasi and Cell Designer have been written for biological modelling.  Uppaal is modelling software for general use, and WEAVE is a general data visualisation tool.

All offered some level of visualisation, some simply graphs, others more complex visualisations.  Bio-PEPA offered only line graphs.  Uppaal had visualisations that highlighted where in a finite state machine view of the model the current state is.  V-Cell had visualisation of the model in a hierarchical set of circles, it could also display spatial elements of the results data by displaying a heat model view of the cell.  Cell-O was aimed more at multi-cell models and was able to show them moving and splitting, it also had visualisation of the model as a finite state machine.  Copasi only graphs although the user had more control over the display of the plots.  WEAVE had the largest visualisation capacity being able to display a variety of standard graph times along with more interesting ones, such as geographical maps, but it did not appear to have anything specialised for biological models.  WEAVE is also the tool that gave the user the most control over the visualisation.

The existing biological modelling software seems to be focused more on the ease of modelling.  The visualisation features on offer are typically quite basic.  They also lack the more innovative features that can be found in the newer general data visualisation software.

As the project scope expanded to include goals not specifically related to visualisation it was prudent to perform another software review covering the new features, in particular software that allowed for collaborative editing.  It is important to see what features are common place, which features are not commonplace but are useful and which features are not useful. This analysis would then be used to guide development of the collaborative features of the new tool.

Three products were chosen for review: Google Drive, Pidoco and Lucid Chart.  Analysis of these software can be found below.

\paragraph{Google Drive} which was previously Google Docs is one of the most widely used collaborative pieces of software by a variety of user types.  The focus of the review was on the word processor.  Of the collaborative software reviewed this was felt to be the most user friendly.  One of the nicest features was a cursor indicating where every user currently editing the document is typing, and each user has a different colour allowing you to know who is doing what in real time.  Many people can edit a single document at once.  As well as editing documents users can also leave comments on the document.  Changes made by users appear near instantly to all users, the speed of editing is very important as it is frustrating as a user to have to wait to see changes others are making.  It is also very easy to invite others to edit the document with you.  Each document has a unique link and if a user visits that link they are taken to the document and can start editing.  Different permissions can be granted to different users allowing some level of collaboration with people who you don't necessarily want to grant full write access, these users can then just look at in real time and offer comments.  Different parts of the editor have different levels of collaboration.  All text that is changed is changed for all, but preferences like font choice are only changed for the user, unless another user edits any text.  Also importantly from a UX perspective is that any conflicts that arise appear to be resolved without any user intervention.  A history of what each user has done to the document is also provided and any unwanted changes can be rolled back.

\paragraph{Pidoco} is a collaborative wireframing tool.  It was not as user friendly for collaboration as Google Drive.  Pidoco is much less instant.  Sometimes manual refreshes were required to display the work the other users had done.  Pidoco also supported multi user editing, however there was no way of seeing what users were editing a document, there was also history of what changes each user had done.  Pidoco has no messaging or commenting system which makes asynchronous collaboration more difficult.  Sharing the work requires an email to be the sent to the user, they cannot simply be given a link.  Again different parts of the workspace have different levels of collaboration.  Any U.I. widgets that are placed are shared, but if one user zooms in on a particular area that other users are not forced to that zoom level.

\paragraph{Lucid Chart} is a collaborative diagramming tool.  Lucid Chart lies between Google Drive and Pidoco in terms of collaboration speed.  It is not as instantaneous as Google Drive, but it does not require periodic manual refreshes like Pidoco.  It also allows multi user collaboration, and documents can be shared by link or email.  Users can be granted read or read and write permissions on the document, like Google Drive, so you can collaborate with people you don't want to be able to fully edit.  It has a chat system and users can comment on the document making asynchronous collaboration easier.  Lucid Chart does not let you see what a user is doing in real time, but it does provide a full revision history so you can see what changes each user has made.  Lucid Chart is different in that it appears to be fully collaborative in that even font preferences get synced across users, if one user clicks bold all users will start typing in bold.

By implementing more advanced visualisation features this tool makes Bio-PEPA a more attractive modelling tool, by bringing the feature set in line with the alternative modelling tools.  There are also features that are not offered by any of the other reviewed biological software -- visualisation and non visualisation features.  None of the other tools gave the users as much control over the look of their results.  Also none of the other tools had the correspondence between the different visualisations that exist in this new tool.  This tool is also the only one that allowed the user to annotate and attach supporting documentation.  None of the other modelling or visualisation tools reviewed offered any sort of collaborative features - real time or not.  This makes this tool unique and innovative.

\subsection{Previous Work}

Early on in the first stage of the project it was decided to separate this project from the Eclipse plugin.  It was felt that Eclipse is not the right tool to do data visualisation in.

The initial development stages were focused on getting the new tool from having zero functionality to matching the visualisation features of the Eclipse plug-in.  This involved writing an early version of the \ac{UI}, parsing the Bio-PEPA results data, and plotting it using matplotlib.

The next stage of the project was to extend the functionality.  The first new feature was intensity plotting where the colour of the line increases in intensity/opaqueness as the population of the species increases.  The next feature added was visualisation of the model.  Model visualisation used a system of nesting circles and rings to build a hierarchical view of the cell from its model components.  Finally the user was given control of the plot, allowing them toggle whether lines are plotted or not, what colour the line is and the thickness of the line.  Figure~\ref{fig:f75_mac_intro} shows how the main screen of the program looked at the end of the first stage of development.

\begin{figure}[h!]
    \centering
    \includegraphics[width=\textwidth]{images/french75_mac.png}
    \caption{Main Screen of the Tool}
    \label{fig:f75_mac_intro}
\end{figure}

Over the course of the first stage of work a number of evaluations were carried out with potential and actual users of Bio-PEPA.  The findings from these evaluations were used to improve the tool.

\subsection{Results}

Break down by project stage?  Aniticipating this to be 1.5 - 2 pages

