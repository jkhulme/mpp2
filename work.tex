\chapter{Work Done}

\subsection{Previous Work}
Could move here from intro, could talk a bit more about how they could be improved.

\subsection{New Work}

Signposting text

For each of the below, why, how inc conceptual problems, impact, how it could be improved

\subsubsection{Annotation}

\paragraph{Annotation of the Graph}

Annotation of the graph was an important feature to add.  It is one of Grinstein's features that data visualisation software should have.  This is because data visualisations should help users analyse their results.  If you had a print out of a graph it would be second nature to draw over it and highlight areas of interest.  This task needs to be able to be supported digitally as well.  As well as helping users analyse their data, being able to annotate also means that images for presentations can be prepared without having to save the graph and open it in an external program.  If you did this and then wanted to change the graph you would have to re-annotate it.  This is frustrating for a user and wastes their time.  Being able to do it from within the visualisation solves this problem as the raw data and the annotation data are together.  Another benefit of being able to annotate is having another way of attaching additional information to the graph when sending it to a colleague. Having this information on the graph saves them from flicking back and forth from an email or other supporting documentation.

Annotation on the graph was relatively straightforward to implement.  matplotlib provides annotation capabilities in two forms.  One form is annotating arrows with or without text, and the other is arbitrary drawing on the graph.

Users of the new tool are provided with four annotation types: arrow, text, arrow with text and circles.  Buttons for each of these annotation types have been placed on the matplotlib toolbar.

There were problems in making annotation user friendly.  Text and circle annotations were intuitive as all they require is one click -- click where you want the annotation and it will be placed there.  However the two arrow annotation types required two clicks.  The first click is the start point (tail of the arrow) and the second click is the finish point (head of the arrow).  This was not obvious, when handed over to the users they didn't know that it required two clicks and didn't know whether the arrows would be drawn head to tail or tail to head.  The technique for placing arrows was changed so that the first click still fixed the position of the tail of the arrow. However the behaviour after the first click has changed, now a temporary annotation is continuously redrawn that has the head of the arrow wherever the mouse is.  This allows the user to see the arrow they are drawing.

Next the annotations had to be able to be edited or deleted.  The annotations can not just be clicked as they are not a \ac{UI} widget like a button.  The solution to this was to have an array of annotations.  When a user right clicks on the graph it searches through all annotations and selects the annotation that was closest to the click (if it was below a certain threshold).  The selected annotation is then highlighted red, and a context menu appears to give feedback to the user that they have successfully selected an annotation.  The context menu then gives the user the option to edit or delete an annotation.  Editing an annotation only allows for editing text.  For changing position the annotation has to be deleted and redrawn.

\paragraph{Annotation of the Animation}

After completing annotation of the graph it was important to expand this to the animation panel, as this is the other area where visualisations are put.  This posed more of a challenge than for annotation of the graph and there were a number of issues to overcome.

\begin{enumerate}
\item How to implement the annotations?  For the graph matplotlib has built in annotation support.  wxPython does have drawing support but not in built annotation support.  Annotating on the animation will need manual handling of the drawing on top of the animation visualisation.  Manual drawing means that the automatic layout functionality that wxPython provides cannot be used.
\item When to display the annotations?  When an annotation is drawn on the graph it is displayed at all times.  The appearance of the graph does not change over time.  However the appearance of the animation visualisation does change over time.  The problem faced when annotating is whether to have annotations available at only specific times in the animation, or to have them there the whole time, and if they are going to appear and disappear how can it be done without being distracting?
\item How to give the user control over the annotations? When a user wants to edit or delete an annotation on the graph it is always there.  However on the animation panel if the annotation is temporal, then it is not always visible for the user to edit or delete and it would be frustrating for a user to constantly have to search through the animation to look for annotations to change them.
\end{enumerate}

\subsubsection{Animation}

THIS NEEDS A LOT OF WORK

Animation was a key goal of the project.  The core aim of the project is to help biologists who aren't comfortable with traditional time-series graphs.  So the goal was to provide them with visualisations closer to what they see in their domain.  This has been accomplished by displaying spatial data in the shape of the cell.

The animation is ideally used to display species moving through a cell.  This collapses what would be multiple different lines in a graph, that give no indication of their real position in the cell, into a single image.  There is one segment in the cell animation for each line.  The colours in each segment reflect the colour of the line on the graph.  Then over the course of the animation the colours are set by the colour in the intensity plot version of the line.  This allows the researcher to compare the two visualisations and will hopefully help build their confidence on the graph plot.

To make setting up the animation user friendly to control required the model file so that the location hierarchy could be parsed.  Before this there was an awkward system where the user had to input where in the cell a species is.  This was time consuming, awkward and quite brittle.  At the time it assumed that there would be three compartments in the species, which is a terrible assumption to make.

The requirement of the model file for parsing animation has also led to animation replacing the previous model visualisation.  In the set up phase after the the model and the species have been parsed the user is presented with a cell segment, similar to what is seen in the animation panel.  The cell segment is split into different regions, one for each region of the cell.  These segments are then coloured if the selected species is present in them.  The user can select between all species in the selected results files.  This has a number of benefits.  First, they can sanity check that they have matching results and model files.  Second, they can see how the model is structured.

Similar control is provided on the animation panel itself.  The user can see the animation focused on a specific species, in which case a cell segment is drawn for each file that the species is in.  Or they can have the animation focused on a specific results file in which case a cell segment is drawn for each species in that results file.

\subsubsection{Data Mining}

\subsubsection{Search}

Using a time series as a plot posed some interesting problems.
\begin{itemize}
\item How to cope with different scales
\item How to cope with events happening at different times
\item How to represent the plot to allow for efficient search
\item How to determine similarity between two graphs
\end{itemize}

All of these needed to be overcome for this feature to be useful.

This is an area of active research.  Early techniques used simple techniques such as Euclidean distance, but these simple approaches gave poor results.  It is feasible to have a species that exhibits a similar reaction to our query (the line has the same shape) but it happens at a different time in the experiment than our query plot.  In other words the graph is offset.  We can tell that the two graphs are similar, but euclidean distance will return that they are unsimilar.  The same problem occurs with differences in species population giving an offset in the x axis.

After researching current techniques an approach was taken that solves all the problems above.

The first step is to convert the input data to be a list of all n length sub lists of the input data. i.e. with input data [1,2,3,4,5] and n = 3 our input data would become [[1,2,3], [2,3,4], [3,4,5]].  The sub lists are our features.  We then normalise each feature to be zero mean, unit variance.

The next step is to convert the continuous data into discrete data and reduce the dimensionality of it.  By normalising the data to be zero mean and unit variance we have allowed a normal distribution to be easily fitted to our data.
\subsubsection{Collaboration}

\subsubsection{Usability}

In all the evaluations of the project users have commented on the difficulty of using parts of the tool.  Action has been taken to make it easier to use.  Many of the changes have been guided by Shneiderman, Norman \& Nielson's guidelines.  Specifics are detailed below.

\paragraph{Undo \& Redo}
Shneiderman calls for easy reversal of actions and Nielson calls for user control and freedom -- an emergency exit from an unwanted state.  To address this, an undo/redo functionality has been added.  This required refactoring of the project code, so that the session data is in one location, inside a singleton. Any changes to this data are picked up during the next \ac{UI} update and are reflected in the visualisations.  The session data is stored as a dictionary.  To implement undo and redo, copies of the data dictionary are pushed and popped onto the stack.  Copies are pushed onto the undo stack on any atomic change the user makes.  This gives the user a full session history to go back through and this was one of the early goals from the first project phase.

A problem was encountered when trying to copy the dictionary onto the stack.  When just pushing the dictionary onto the stack it would not put a new copy of it onto the stack, so any changes to the dictionary after it has pushed onto the stack are also there in the stack.  Python dictionaries have a copy method.  Copy only does a shallow copy -- any objects in the original dictionary will have their reference placed in the new dictionary.  This was fine for some parts of the session dictionary, but for others it was not. In particular, lines and annotations, which are custom objects presented problems.  This was solved by using deep copy.  With deep copy a new copy is made of objects as well.  Some elements of wxPython and matplotlib were unable to be deep copied, but this was fixed whilst focusing more around the data -- so the \ac{UI} elements use the data, not the other way around.

\paragraph{Saving}

It is important that a user is able to save and load the visualisation session as they may not be able to complete all their analysis in one sitting and may want to come back to their work in the future.  Without the ability to save and load the user would have to repeatedly add annotations and change preferences and attach files.  It was possible to add Saving and loadingby building on the work done to implement undo \& redo, although further work was required. Python has a module called pickle to serialise and deserialise data.  When saving, the dictionary containing the session data is pickled and written to a file and when loading the reverse happens.  Because the program is now focused on the data model, once a previous session has been loaded, a \ac{UI} refresh is triggered and the visualisation reflects the loaded data.

Saving the data also enables limited collaboration.  The user can customize the appearance and add annotations.  They can then save the state to a file and email that file to a colleague.  The colleague can then load the file and see the user's work.  The colleague can then correct any issues and add their own work.  The colleague can then save this and email it back to the user.  This is useful and is better than no collaboration, but it is entirely non realtime.

\paragraph{Feedback}
Norman and Shneiderman both call for feedback to be given to the user so that the user can be sure that an action has been accomplished.  This feedback can come in a number of different forms and was in place in some parts of the project already.

Existing feedback in the project was a natural byproduct of some of the features; for example, when loading a results file the feedback that the load operation has been successful is that a graph appears on the screen. If the graph does not appear then something has gone wrong.  Additional feedback has been added to the project:
\begin{itemize}
\item When adding annotations the cursor changes to indicate to the user that they can interact with the graph in a different way.
\item The title bar text changes to display ``unsaved'' when the user makes a change and then changes back to ``saved'' when a successful save has been performed.
\end{itemize}

\paragraph{Guiding the User}

The first evaluation of the second phase of the project unearthed that the users struggled to choose the correct action as there were multiple ways of doing the same action that had slightly different use cases.  There was also confusing language in the menu options.  These multiple paths have been removed. For example, now there is only one way to open results files initially.  To help guide the user further \ac{UI} elements are enabled and disabled as appropriate.  Now when the program is first loaded the only action a user can perform is to load a session or start a new session.  Afterwards other \ac{UI} elements are enabled to allow the user to start using the tool effectively.

\subsubsection{Data Manipulation and Export}

\subsection{Finished Product}
Overview and walkthrough of tool
